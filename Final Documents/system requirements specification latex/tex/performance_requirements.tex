
\subsection{Operating time for the camera rig shall have a minimum of one hour of battery life}
\subsubsection{Description}
All the individual components of the camera rig and the amount of data to be transferred will take most of the power from the camera rig. The goal of the camera rig is to send a clear video feed from a location back to the virtual reality headset. So, a battery life of a minimum of one hour will be plenty for this project.
\subsubsection{Source}
CSE Senior Design project specification
\subsubsection{Constraints}
The size of the camera rig will have an impact on the amount of power needed to operate the camera rig.
\subsubsection{Standards}
None
\subsubsection{Priority}
Critical Priority: Without a decent amount of battery power, the camera rig will not operate as intended or be inoperable.

\subsection{Video latency shall be under 50ms}
\subsubsection{Description}
The video being streamed from the camera device to the virtual reality headset must be delivered within a 50 millisecond timeframe.
\subsubsection{Source}
CSE Senior Design project specification
\subsubsection{Constraints}
Budgetary constraints may result in hardware with some latency.
\subsubsection{Standards}
None
\subsubsection{Priority}
Moderate Priority: User comfort is somewhat reliant on the latency of the video.

\subsection{Video shall have 60 frames per second video rendering}
\subsubsection{Description}
In order for the video to be believable, minimum video specifications must be met. Based on developer opinions and video enthusiast message boards, the minimum agreed upon specifications are 720p at 30fps. The group believes that these specs are more than achievable, and aims to produce a product that delivers a reliable 1080p at 60fps.
\subsubsection{Source}
CSE Senior Design project specification
\subsubsection{Constraints}
Budget constraints may limit the capabilities of the components available during development.
\subsubsection{Standards}
None
\subsubsection{Priority}
High Priority: To experience virtual reality in real-time, rendering high quality video is important.

\subsection{Video shall be stable when camera rig is in motion up to 5 MPH}
\subsubsection{Description}
The video feed from the camera rig to the virtual reality headset should be stable when the camera rig is in motion of no more than 5 miles per hour. The operation of the camera rig is not meant to be used under extreme conditions regarding to speed because of the amount of data that has to be transferred to the virtual reality headset.
\subsubsection{Source}
CSE Senior Design project specification
\subsubsection{Constraints}
The type of algorithm used will affect the video feed.
\subsubsection{Standards}
None
\subsubsection{Priority}
High Priority: If the camera rig is moving higher than 5 mph, video feed will be unstable, and it will possibly crash.

\subsection{The camera rig shall have a maximum translation error of 0.5 degrees}
\subsubsection{Description}
The camera rig will match the user's head movement to a tolerance of 0.5 degrees.
\subsubsection{Source}
CSE Senior Design project specification
\subsubsection{Constraints}
Software and hardware capabilities may affect the translation error.
\subsubsection{Standards}
None
\subsubsection{Priority}
Moderate Priority: One-to-one movement between the camera device and the user's virtual reality headset makes a massive difference in the user's experience with the device.
