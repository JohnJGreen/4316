A stereo camera is a type of camera with two or more lenses with a separate image sensor for each lens. The primary use for a stereoscopic camera comes from the ability to capture three-dimensional images. Stereo cameras may be used for making stereoviews and three-dimensional pictures for movies, or for range imaging. The three-dimensional aspect of the stereo camera is achieved by separating the cameras by a fixed distance. The separation provides a sense of depth when the images are combined. The distance between the lenses in a typical stereo camera is about the distance between one's eyes and is about two and a half inches. The stereo camera has been in use since the 1940's with the Realist format cameras. The technology has progressed significantly over the years. The proliferation of computer vision applications has transformed stereo cameras from an artistic form to a more industrial use. A stereo camera combined with a computer vision algorithm provides the ability to accurately gauge distance and generate depth maps. On the artistic front, the introduction of 3D movies relies entirely on stereo cameras for the non-computer generated content.

A virtual reality headset is comprised of a stereoscopic head-mounted display, providing separate images for each eye, and head motion tracking sensors. The idea of a virtual or augmented reality has motivated attempts to develop a commercial vritual reality headset as early as 1994, Forte VFX1. The goal of creating an immersive virtual reality experience for the wearer has been hindered by the lack of computational power needed to drive such a device. The cost to develop and operate a virtual reality headset limited the potential applications to simulators and training facilities. This cost limitation changed in 2012 with the funding campaign that began the Oculus Rift. Over a four year period, many developments were made in the consumer virtual reality headset market. The concurrent development of the HTC Vive to compete with the Oculus has left the consumer market with two functional and cheap virtual reality headsets.